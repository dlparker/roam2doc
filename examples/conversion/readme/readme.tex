% Intended LaTeX compiler: pdflatex
\documentclass[11pt]{article}
\usepackage[utf8]{inputenc}
\usepackage[T1]{fontenc}
\usepackage{graphicx}
\usepackage{longtable}
\usepackage{wrapfig}
\usepackage{rotating}
\usepackage[normalem]{ulem}
\usepackage{amsmath}
\usepackage{amssymb}
\usepackage{capt-of}
\usepackage{imakeidx}
\makeindex[intoc]
\usepackage{times}
\usepackage{hyperref}
\hypersetup{
  colorlinks=true
}
\author{dparker}
\date{\today}
\title{roam2doc parse of /home/dparker/projects/roam2doc/README.org}
\setcounter{secnumdepth}{6}
\setlength{\parindent}{0pt}
\setcounter{tocdepth}{6}
\begin{document}
\maketitle
\tableofcontents
\clearpage
\section{A set of python tools for convering org-roam files to a single document \index{ A set of python tools for convering org-roam files to a single document} }
 \label{obj-3}
 \label{obj-2}
\begin{enumerate}
\item
Can parse single or multiple org files, detecting and handling any roam links.
\item
Converts parsed org data into a tree structure, which is available for output in json format.
\item
Converts tree structure to html, preserving a large subset of the visual structure of
   the original files, and converting both internal and roam links into html links within
   the converted document

\item
Converts tree structure to latex, with with all the same features as the html version, except
   for differences in how document structures work in the two formats.

\item
Optionally converts the html to pdf using wkhtml2pdf or letex to PDF using pdflatex.
\item
Focused on the document creation aspect of org and org-roam, not on the todo lists, schedules, etc.
\item
Supports custom block surrounded with keywords
\texttt{\#+BEGIN\_FILE\_INCLUDE}
\sout{BEGIN\_FILE\_INCLUDE\textasciitilde{} and \textasciitilde{}\#}
\texttt{\#+BEGIN\_FILE\_INCLUDE}
  where each line in the block is treated as a file path, either full path starting at / or a path
  relative to the file that contains the include block. If the path resolves then whatever it
  contains replaces the include blog. Hopefully the contents are in org mode format, or all bets are off.

\item
On request, it generates a PDF that has special markup in the text and a cross reference at the end
  of the generated document that is designed to enable an AI (tested with Grok3) to figure out the
  internal links in the file after it has passed through OCR. This is particularly important if you
  are making the doc from a collection of org-roam files, as they tend to be writen as topic notes
  with minimal context and much of the available context arrises from the links between them. 
  For a look at how the cross reference and annotations work you can read grok's note to future
sessions in
\textbackslash{}textit\{!!! link target "file:examples/plain/note\_from\_grock.org" not found !!!\}
\vspace{\baselineskip}

\end{enumerate}
\section{HOW to use it \index{ HOW to use it} }
 \label{obj-47}
 \label{obj-46}
\begin{enumerate}
\item
Until this becomes and actual package that includes an exectuable:
\begin{enumerate}
\item
PYTHONPATH=.
\emph{src src}
roam2doc --help
\end{enumerate}
\item
There are two options for producing PDF, both of which require external tools
\begin{enumerate}
\item
The preferred method is to use the latex to pdf tool
      "pdflatex". This is the preferred way because it generates a decent
      table of contents and index, and can produce special markup for AIs
      (using the -grokify switch). The generated latex output uses various latex
      packages, all of which are available on ubuntu like this:

\begin{enumerate}
\item
sudo apt update
\item
sudo apt install texlive-full textlive-latex-extra
\end{enumerate}
\item
Alternatively you can generate a pdf from html using wkhtmltopdf. It
      currently does not produce and index or the cross reference for AIs,
      but it does look nicer, at least to me. I may eventually add an index
      (if wkhtmltopdf supports it) and the AI cross reference (that is just work)

\end{enumerate}
\item
If you want to use wkhtmltopdf, then you
   will want to ensure that you have the patched QT version of wkhtml2pdf. The
   unpatched version will not handle links properly, nor produce a table of contents.

\item
See some examples in action
\begin{enumerate}
\item
To see the result of combining roam files:
\begin{verbatim}
PYTHONPATH=./src src/roam2doc --help examples/roam/roam1/roam_combine1.list -o roam1.html --overwrite --doc_type=html
or
PYTHONPATH=./src src/roam2doc --help examples/roam/roam1/roam_combine1.list -o roam1.latex --overwrite --doc_type=latex
or 
PYTHONPATH=./src src/roam2doc --help examples/roam/roam1/roam_combine1.list -o roam1.pdf --overwrite --doc_type=pdf --grokify

\end{verbatim}
\item
To see how the include mechansim works
\begin{verbatim}
PYTHONPATH=./src src/roam2doc --help examples/roam/roam2/roam_combine2.list -o roam2.pdf --overwrite --doc_type=pdf --grokify
\end{verbatim}
\item
To see the result of parsing a large number of org content types:
\begin{verbatim}
PYTHONPATH=./src src/roam2doc --help examples/plain/all_nodes.org -o all.html --overwrite    
\end{verbatim}
\item
To see the result of parsing this file:
\begin{verbatim}
PYTHONPATH=./src src/roam2doc --help README.org -o readme.html --overwrite    
\end{verbatim}
The full help:

\begin{verbatim}
usage: cli.py [-h] [-o OUTPUT] [-t {html,json,latex,pdf}] [-j] [-g] [-l {error,warning,info,debug}] [--overwrite] [--wk_pdf] input

Convert org-roam files to HTML documents.

positional arguments:
input                 Input file (.org), directory containing .org files, or file list with paths

options:
-h, --help            show this help message and exit
-o OUTPUT, --output OUTPUT
Output file path for HTML (default: print to stdout)
-t {html,json,latex,pdf}, --doc_type {html,json,latex,pdf}
Output file path for HTML (default: html)
-j, --include_json    Include a json version of the parsed document tree in the html head section
-g, --grokify         Produce a link cross reference table in pdf suitable for AI input
-l {error,warning,info,debug}, --logging {error,warning,info,debug}
Enable logging at provided level, has no effect if output goes to stdout
--overwrite           Allow overwriting existing output file (default: False)
--wk_pdf              Use wkhtmltopdf to convert output to PDF

\end{verbatim}
\vspace{\baselineskip}
\end{enumerate}
\end{enumerate}
\section{Things to know \index{ Things to know} }
 \label{obj-103}
 \label{obj-102}
\vspace{\baselineskip}
\subsection{Things it does that might surpise you \index{ Things it does that might surpise you} }
 \label{obj-107}
 \label{obj-106}
\begin{enumerate}
\item
Org Keyword strings are stripped from the text during parsing. The only keyword that has
  any effect is the \#+NAME: keyword, which (if at line beginning) is applied to the next
  non-keyword line. This allows you to name an element (e.g. a table) and then link to
  it by name

\end{enumerate}
\subsection{Things it doesn't do and probably should \index{ Things it doesn't do and probably should} }
 \label{obj-117}
 \label{obj-116}
\begin{enumerate}
\item
Footnotes are not parsed, they will be treated as ordinary text
\item
Drawers that are either property drawers at the beginning of a file or are property drawers for
  heaading are parsed, all other drawers are not parsed, just treated as text.

\item
Verbatim strings cannot contain equal sign "=", use \textasciitilde{} (inline code) if you need that in your text.
\end{enumerate}
\subsection{Things it doesn't do and maybe never will \index{ Things it doesn't do and maybe never will} }
 \label{obj-129}
 \label{obj-128}
\begin{enumerate}
\item
Parse and do something useful with the time management aspects of org files, this
\item
Inlinetasks are not parsed, they will be treated as headings an will make things ugly
\end{enumerate}
\subsection{Things it doesn't do and probably won't \index{ Things it doesn't do and probably won't} }
 \label{obj-137}
 \label{obj-136}
\begin{enumerate}
\item
Run wkhtml2pdf on windows. Works on linux, will probably work on Mac
\item
Produce LaTex output, including any LaTex features found in the org files
\end{enumerate}
\subsection{Things that might be nice to add someday \index{ Things that might be nice to add someday} }
 \label{obj-145}
 \label{obj-144}
\begin{enumerate}
\item
Provide option to allow uset to supply css and or javascript contents to
  be merged into the head of the html output. There is already an option
  to include a json object version of the parsed tree into the head, so
  you could write code to inspect that object and do interesting things.
  Of course you can do this just by editing the output directly.

\end{enumerate}
\subsection{History, what I wanted and why it lead to this. \index{ History, what I wanted and why it lead to this.} }
 \label{obj-156}
 \label{obj-155}
\subsubsection{What for? \index{ What for?} }
 \label{obj-159}
 \label{obj-158}
  I wanted to be able to take notes on a wide range of topics and relate them together
  into a book outline. Orgroam perfectly fit my style, so I started learning it.

\subsubsection{First problem \index{ First problem} }
 \label{obj-165}
 \label{obj-164}
  I had also just started using the Grok3 AI to work on the research I was turning into notes,
  so I wanted to be able to load all the notes into the Grok context before submitting
  prompts. Grok informed me that orgroam files would not work as well as I wanted because
  it wouldn't do well interpreting the org files, and especially the links. Grok suggested
  that I would get much better results if I could collect the files into single document
  such as a PDF. So I needed a tool to do this. I prefer to look for python based solutions
  to such problems since I can modify or extend them if I need to, python being my favorite
  language.

\subsubsection{The First Fix \index{ The First Fix} }
 \label{obj-177}
 \label{obj-176}
I found the pyorg package at
\href{https://github.com/nasa9084/py-org}{https://github.com/nasa9084/py-org}
.
  Its main purpose was to export org content to html, and I have experience using
  wkhtmk2pdf to create PDFs, so that seemed workable. I forked to
\href{https://github.com/dlparker/pyorg2}{https://github.com/dlparker/pyorg2}
and was able to quickly modify it to add support
  for roam links.
  I got Grok to help me by updating the tests from nodetest to pytest.
  I then upgraded the tests to get 100\% coverage. Seemed like a good start

\subsubsection{Now I have two problems \index{ Now I have two problems} }
 \label{obj-197}
 \label{obj-196}
  As I started looking at how I wanted to use this, it became clear that I also wanted to
  support org internal links, which the orignal package did not. The linking to something
  part is simple, but the range of link targets that org supports lead to some complexity
  when thinking about adding it to the package. For example, you can link to a Table and
  almost any other element of an org file but giving it a name using a \#NAME+ keyword like so:

\begin{verbatim}
#+NAME: my_table
| col 1       | col 2           |
| row 1 col 1 | row 1 col 2     |

[[my_table][link to my table]]  
\end{verbatim}
\vspace{\baselineskip}
  Also adding complexity to the needed changes is the fact that a link/reference can
  appear in many places other than plain text. Inside table cells, for example.

  The original package's parsing had some other limitations as well, which may well have
  been the author's intention to keep the task at hand to a useful limited subset of org
format. The full format is pretty rich. See
\href{https://orgmode.org/worg/org-syntax.html}{https://orgmode.org/worg/org-syntax.html}

\vspace{\baselineskip}
  The scale of the modifications needed to achieve my goals convinced me that I was going
  to contort the structure so badly that it would be dificult to maintain. So I decided
  to start over.
\vspace{\baselineskip}

\printindex
\end{document}